%%%%%%%%%%%%%%%%%
% This is an sample CV template created using altacv.cls
% (v1.1.5, 1 December 2018) written by LianTze Lim (liantze@gmail.com). Now compiles with pdfLaTeX, XeLaTeX and LuaLaTeX.
%
%% It may be distributed and/or modified under the
%% conditions of the LaTeX Project Public License, either version 1.3
%% of this license or (at your option) any later version.
%% The latest version of this license is in
%%    http://www.latex-project.org/lppl.txt
%% and version 1.3 or later is part of all distributions of LaTeX
%% version 2003/12/01 or later.
%%%%%%%%%%%%%%%%

%% If you need to pass whatever options to xcolor
\PassOptionsToPackage{dvipsnames}{xcolor}

%% If you are using \orcid or academicons
%% icons, make sure you have the academicons
%% option here, and compile with XeLaTeX
%% or LuaLaTeX.
% \documentclass[10pt,a4paper,academicons]{altacv}

%% Use the "normalphoto" option if you want a normal photo instead of cropped to a circle
% \documentclass[10pt,a4paper,normalphoto]{altacv}

\documentclass[10pt,a4paper,ragged2e]{altacv}

%% AltaCV uses the fontawesome and academicon fonts
%% and packages.
%% See texdoc.net/pkg/fontawecome and http://texdoc.net/pkg/academicons for full list of symbols. You MUST compile with XeLaTeX or LuaLaTeX if you want to use academicons.

% Change the page layout if you need to
\geometry{left=1cm,right=9cm,marginparwidth=6.8cm,marginparsep=1.2cm,top=1.25cm,bottom=1.25cm}

% Change the font if you want to, depending on whether
% you're using pdflatex or xelatex/lualatex
\ifxetexorluatex
  % If using xelatex or lualatex:
  \setmainfont{Carlito}
\else
  % If using pdflatex:
  \usepackage[utf8]{inputenc}
  \usepackage[T1]{fontenc}
  \usepackage[default]{roboto}
  %\usepackage[lf, sfdefault]{gandhi}
  %\usepackage[default]{lato}
  %\usepackage[ttdefault=true]{inconsolata}
  \usepackage{inconsolata}
\fi

% Change the colours if you want to
\definecolor{Mulberry}{HTML}{72243D}
\definecolor{customgold}{HTML}{a47000}
\definecolor{SlateGrey}{HTML}{2E2E2E}
\definecolor{LightGrey}{HTML}{666666}
\colorlet{heading}{SlateGrey}
\colorlet{accent}{customgold}
\colorlet{emphasis}{SlateGrey}
\colorlet{body}{LightGrey}

% Change the bullets for itemize and rating marker
% for \cvskill if you want to
\renewcommand{\itemmarker}{{\small\textbullet}}
\renewcommand{\ratingmarker}{\faCircle}

%% sample.bib contains your publications
%\addbibresource{sample.bib}

\usepackage{multicol}

\begin{document}
\name{Aaron Hadley}
\tagline{Undergraduate student seeking a full time position \\in research, cybersecurity, or software engineering }
\personalinfo{%
  % Not all of these are required!
  % You can add your own with \printinfo{symbol}{detail}
  \email{aahadley1@gmail.com \quad }
  \location{Orlando, Florida} \hfill\\
  \github{github.com/aahadley \quad \ \ }
  \phone{(407) 401-3174}
  %% You MUST add the academicons option to \documentclass, then compile with LuaLaTeX or XeLaTeX, if you want to use \orcid or other academicons commands.
  % \orcid{orcid.org/0000-0000-0000-0000}
}

%% Make the header extend all the way to the right, if you want.
%\begin{fullwidth}
\makecvheader
%\end{fullwidth}

%% Depending on your tastes, you may want to make fonts of itemize environments slightly smaller
\AtBeginEnvironment{itemize}{\small}

%% Provide the file name containing the sidebar contents as an optional parameter to \cvsection.
%% You can always just use \marginpar{...} if you do
%% not need to align the top of the contents to any
%% \cvsection title in the "main" bar.

\cvsection[page1sidebar]{Education}
    
\cvevent{Bachelor of Science: Computer Science \quad GPA: 3.1 \\ Minor: Mathematics}{University of Central Florida}{Expected December 2019}{Orlando, FL}
    
\begin{multicols}{2}
    \textbf{Undergraduate Coursework:}
    \begin{itemize}
        \item Cryptography
        \item Computer Architecture
        \item Operating Systems
        \item Digital Forensics
    \end{itemize} \vfill\null
    \columnbreak
    \textbf{Graduate Coursework:}
    \begin{itemize}
        \item Quantum Computing
        \item Quantum Information Theory
        \item Advanced Engineering \\Mathematics
    \end{itemize} \vfill\null
\end{multicols}

\cvsection{Research}
    
\cvevent{Theorem Proving with Deep Reinforcement Learning}{UCF Computer Science Department}{January 2019 -- Ongoing}{}
\begin{itemize}
    \item Investigating deep learning techniques for automated theorem proving software.
    \item Using Microsoft's Lean prover to interact with a deep learning interface.
\end{itemize}

% TODO: Rework bullet-points to be more quantitative.
    \divider
    
\cvevent{Quantum Information Group}{UCF Physics Department}{October 2018 -- Ongoing}{}

\begin{itemize}
    \item Participating in weekly journal clubs discussing advanced topics in:
    
    \begin{itemize}
        \item Simulation of quantum systems
        \item Condensed matter physics
        \item Quantum communications
    \end{itemize}
    
    \item Preparing regular presentations on quantum cryptography and \\key distribution.
\end{itemize}


\cvsection{Projects}
    
\cvevent{RandomBag.net}{Personal - KnightHacks 2019}{}{}
\begin{itemize}
    \item Wrote a quantum random number generator using IBM’s quiskit platform
    \item Built a flask app that used Zinc API, with the QRNG to fill an Amazon cart with random items.
\end{itemize}
    \divider
    
\cvevent{Richard Leinecker Appreciation Application: The Maze}{Coursework}{}{}
\begin{itemize}
    \item Worked with a team of 9 to develop an application that allows many users to simultaneously interact with a Unity game.
    \item Built an API to take commands from a mobile app and deliver them to a Unity application.
\end{itemize}


\medskip
\clearpage

%% If the NEXT page doesn't start with a \cvsection but you'd
%% still like to add a sidebar, then use this command on THIS
%% page to add it. The optional argument lets you pull up the
%% sidebar a bit so that it looks aligned with the top of the
%% main column.
%\addnextpagesidebar[-1ex]{page1sidebar}


\end{document}
